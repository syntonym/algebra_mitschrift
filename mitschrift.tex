\documentclass[12pt,a4paper,ngerman]{scrreprt}
\usepackage{ngerman,amssymb,amstext, amsmath}
\usepackage{enumerate}

\newcommand{\gdw}{\Leftrightarrow}

\begin{document}
\title{Algebraische Strukturen}
\subtitle{Mitschrift}
\author{Simon Haendeler}
\maketitle
%\tableofcontents
%TODO Sinnvolle Gliederung für den Inhalt (welche Sachen werden in das Inhaltsverzeichnis aufgenommen)
\setcounter{chapter}{-1}
\chapter{Allgemeine Definitionen und Notationen}
Wir setzen den Begriff der Menge vorraus.

Seien $X$, $Y$ Mengen. Eine Abbildung $F: X \to Y, x \mapsto y = f(x)$ ist eine "`Zuordnung"' die jedem Element $x \in X$ genau ein $y \in Y$ zuordnet.
Genauer: $\Gamma(f)$ von $f$ als Teilmenge des kartesischen Produktes $\Gamma(f) \subset X \times Y$ mit folgender Bedingung $\forall x \in X \exists _1  y \in Y$ mit $(x,y) \in \Gamma(f)$.

\section{Bemerkung:}
Vielfach werden Objekte in der Mathematik konstruiert, oder durch ihre Eigenschaften eindeutig festgelegt. Axiomatische Defintion des Kartesischen Produktes von Mengen durch die Universelle Eigenschaft:


Seien $X, Y$ zwei Mengen. 
Eine Menge $Z$ gemeinsam mit Abbildung
\begin{enumerate}[]
\item $\pi: Z \to X$ ($\widehat{=}$ Projektion auf den ersten Faktor) und 
\item $\tau: Z \to Y$ ($\widehat{=}$ Projektion auf den zweiten Faktor)
\end{enumerate}
heisst kartesisches Produkt von $X$ mit $Y$ wenn folgende Universelle Eigenschaft erf"ullt ist: F"ur alle Mengen $W$ mit Abbildung $\alpha: W \to X$ und $\beta: W \to Y$ existiert folgendes kommutative Diagramm.

%TODO kommutatives Diagramm einf"ugen

\section{Bemerkung:}
\begin{enumerate}[(a)]
\item Wenn ein kartesisches Produkt exestiert (bei Mengen ist dies der Fall), so ist es eindeutig bis auf eine eindeutige Isomorphie (ohne Beweis). 

\item Die Menge der Paare $X \times Y$ zusammen mit den beiden Projektionen $\pi X \times Y \to X (x,y) \mapsto x$ und $\tau X \times Y \to Y (x,y) \mapsto y$ erf"ullt die universelle Eigenschaft.
\end{enumerate}

\section{Definition Verkn"upfung:}

Sei $X$ eine Menge und $X \times X$ das kartesische Produkt von $X$ mit sich selbst.
Eine (bin"are) Verkn"upfung auf $X$ ist eine Abbildung 
$\mu: X \times X \to X (x,y) \mapsto \mu(x,y) = z$

\section{Beispiele}

\begin{enumerate}[(1)]
\item Addition oder Multiplikation auf $\mathbb{Z}$ oder $\mathbb{R}$ \\ $\mu(x,y) = X+Y$\\
\item Die Matrizenmultiplikation auf $GL_n(K)$, K K"orper \\ $\mu(A,B) = A*B$\\
\item Kompositionen von Endomorphismen eines K"orper-Vektorraumes $V$.\\
$f,g \in End_K(V) = \{ h:V\to V | $h lineare Abbildung $\}$\\
$\mu (f,g) = f \circ g = fg$\\
\end{enumerate}

\section{Definition Verkn"upfungsgebilde:}

Ein Verkn"upfungsgebilde ist eine Menge $X$ zusammen mit einer Verkn"upfung
$\mu: X \times X \to X $ geschrieben $X = (X\mu)/(X,+)/(X,\circ)/(X,\times)/(X,*)$

\section{Bemerkung:}

Eine Menge kann auch mehrere Verkn"upfungen besitzen, etwa $(\mathbb{Z},+)$ und $(\mathbb{Z},.)$

Das Gebiet Algebra ist das Studium von Verkn"upfungsgebilden mit speziellen Eigenschaften. Entsprechend gibt es verschieden Begriffe/Objekte, wobei auch "aussere Verkn"upfungen zugelassen werden
bzgl. einer zweiten Menge $R$, gegeben durch eine Abbildung $\lambda: R \times X \to X; (r,x) \mapsto \lambda(r,x)$ (Skalarmultiplikation)

Die wichtigsten Objekte in der Algebra sind:

\begin{enumerate}[]

\item Jeweils eine Verkn"upfung:

\begin{enumerate}[]
\item Monoide
\item Gruppe
\end{enumerate}

\item Jeweils zwei innere Verkn"upfungen:

\begin{enumerate}[]
\item K"orper
\item Ringe
\end{enumerate}

\item Jeweils eine innere und eine "aussere Verkn"upfung bzgl. eines Ringes oder eines K"orpers

\begin{enumerate}[]
\item Vektorr"aume
\item Moduln
\end{enumerate}

\item Drei Verkn"upfungen: zwei innere, eine "aussere

\begin{enumerate}[]
\item Algebren
\end{enumerate}

\end{enumerate}

\section{Definition:}

Abbildungen zwischen solchen Algebraischen Objekten/algebraischen Strukturen heissen Homomorphismen wenn sie mit der gegeben Verkn"upfung vertr"aglich sind.

%Zweite Vorlesung: 7.8.2013

\section{Definition "Aquivalenzrelation}
Sei $X$ eine Menge. Eine "Aquivalenzrelation auf $X$, geschrieben $\sim$ oder $x \sim y$, ist eine Teilmenge $R \subset X\times X$ , bestehend aus den Paaren $(x,y)$ mit $x \sim y$, die folgende Bedinungen erf"ullt:
\begin{enumerate}
\item reflexiv: $\forall x \in X: x\sim x$ (oder: $(x,x) \in R$)
\item symmetrisch: gilt $x\sim y$ so auch $y \sim x$ \\ (oder: $(x,y) \in R \gdw (y,x) \in R$)
\item transitiv: $\forall x,y,z$ in $X$ gilt $x \sim y$ und $y \sim z$ so auch $x \sim z$ \\ (oder auch: $(x,y),(y,z) \in X \Rightarrow (x,z)\in R$)
\end{enumerate}
Ist $x \in X $, so heisst $\bar{x} = [x] = {y \in X, y\sim x}$ die "Aquivalenzklasse (Restklasse) bez"uglich~$\sim$. Die Menge aller "Aquivalenzklassen wird mit $X/\!\sim  = 
X /R$ bezeichnet.
%TODO Menge aller "Aquivalenzklassen sch"oner machen

\section{Beispiel}
%TODO Besser formatieren

Sei $m \in \mathbb{Z}$. Definiere $x \sim y :\gdw x-y \in m \mathbb{Z} \gdw m$ teilt $x-y$
Schreibweise $\mathbb{Z}/\sim = \mathbb{Z}/m\mathbb{Z} = \mathbb{Z}_m = \{\bar{0}, \bar{1}, ..., \bar{m-1}\}$

\section{Bemerkung:}
%TODO Besser formatieren (Absätze)

Ist $\sim$ eine "Aquivalenzrelation auf $X$, und sei $f: X \to Y$ eine Abbildung, dann existiert eine Induzierte Abbildung $\bar{f}: X/\sim \to Y$ mit kommutativem Diagramm 
%TODO Menge aller "Aquivalenzklassen sch"oner machen
%TODO Kommutatives Diagramm einf"ugen
genau dann, wenn $\forall x, y \in X $ mit $x \sim y$ gilt: $f(x) = f(y)$.

(Also: "Aquivalente Elemente haben das gleiche Bild unter $f$)

Man sagt dann: $f$ faktorisiert "uber $\sim$ oder "uber $X/\!\sim$. Falls $\bar{f}$ exestiert, ist es eindeutig und gegeben durch: $\bar{f}(\bar{x}):=f(x)$

\section{Beweis}
%TODO Der Beweis ist schlecht formatiert. Insb. Einr"uckungen, Absätze und Abstände
%TODO Latexzeichen für := und =: einsetzen

$\Rightarrow$

Sei $\bar{f}: X/\!\sim \to Y$ gegeben und seien $x,y \in X$ mit $x\sim y$. Zu Zeigen: $f(x) = f(y)$
Aber: Da Diagramm kommutiert gilt:
$f=\bar{f}\circ \pi$, also $f(x)=\bar{f} \circ \pi(x) =$
$\bar{f}(\bar{x}) = \bar{f}(\bar{y}) = \bar{f}(\pi(y)) = \bar{f} \circ \pi(y) = f(y)$

$\Leftarrow$

Es gelte $f(x) = f(y)$ f"ur alle $x,y \in X$
%TODO Das hier ist komisch. Es exestiert bar{f} ohne eigenschaften
Zu Zeigen $\exists \bar{f}$
Definiere $\bar{f} X/\!\sim \to Y$ durch $\bar{x} \mapsto f(x)$

Ausf"urlich:

%TODO Hier stimmt irgendwie was nicht so ganz...
$\bar{x} = \{z\in X, z \sim x \} \subseteq X $Teilmenge$ =: Ax$
$X/\sim = \{Ax | x\in X\}$ und $\bar{f}(Ax) := f(x)$ durch die Wal eines Elementes $x \in Ax$
Achtung: Dieses x ist nicht eindeutig durch Ax gegeben, denn 
$z \sim x \Leftrightarrow Ax = Az$
z.z. Definition von $\bar{f}$ ist wohldefiniert, also 
unabh"angig von der Wahl des Repr"asentanten.
Damit ist zu zeigen $x \sim z \Rightarrow f(z) = f(x)$
Dies ist aber gerade die Vorrausetzung im Lemma. 
Finde dazu Beispiele!


\chapter{Gruppen:}
\section{Beispiele:}
\begin{enumerate}[]
\item (\{1\}, *),  (\{0\},+) Triviale Gruppen
\item $(\mathbb{Z},+),(\mathbb{Q},+), (\mathbb{R}, +), (\mathbb{C}, +)$
\item $(\mathbb{Q}^{*}, *) = (\mathbb{Q}$\textbackslash$\{0\}, *), (\mathbb{R}^*,*), (\mathbb{C}^*,*), (S^1, *), (\{\pm 1\},*)$
\item $(\{\sqrt[k]{1}\}, *)$ kte Einheitswurzeln, $k \in \mathbb{N}$
\item $(GL_n(k), *) \forall $ K"orpe, $(SL_n(K), *), (O_n (\mathbb{R}),*), (U_n(\mathbb{C}), *)$
\item $(S_n,\circ)$ Permutationen "uber 1...n Elemente.
\end{enumerate}

\section{Gegenbeispiele:}

$(\{1\},+),(\mathbb{Z},*),(\mathbb{R},*),(\mathbb{N}+),(M_n(\mathbb{R}),*), (Enx(x), \circ), Pot(X), \cup ), (\mathbb{R}, sup),$ ...

\section{Definition Gruppe}

Eine Gruppe ist eine Menge $G$ zusammen mit einer bin"aren Verkn"upfung $G\times G \to G$ : $ (x,y)\mapsto x \circ y$
(meistens Addition oder Multiplikation oder Komposition), sodass folgende Axiome erf"ullt sind:

\begin{enumerate}[({G}1)]
\item Assoziativit"at: $(x\circ y) \circ z = x \circ (y \circ z) \forall x,y,z \in G$
\item Neutrales Element: $\exists e = 1 = 1_G = \mathbb{1} = e_G \in G x \circ e = x = e \circ x \forall x \in G$
\item Inverses Element: $\forall x \in G \exists y \in G$ mit $ x \circ y = y \circ x = e$ Schreibweise: $y = x^{-1}$
\\ $G$ heisst kommutativ wenn zus"atzlich gilt:
\item Kommutativit"at $ x \circ y = y \circ x \forall x,y \in G$ (auch: $G$ ist abelsch)
\end{enumerate}

\section{Bemerkung}
%TODO Entscheiden ob die kurzen Rechnungen in einzelnen Zeilen bleiben sollen.
\begin{enumerate}[(a)]
\item Schreibweise $x \circ y = x+y / x*y = xy / x \circ y$
\item Das Neutralelement ist eindeutig. Seien $e,e'$ zwei neutrale Elmente, dann: 
$$ e = ee' = e'$$ da $e$ und $e'$ neutral

\item Das inverse Element ist auch eindeutig
$$e^{-1} = e$$
$y,z \in G$ Inverse zu x, dann $$y=ye = y(xz) = (yx)z = ez = z$$

\item $y \in X$ heist Rechtsinverse zu $x \in X \Leftrightarrow x \circ y = e$ Linksinverse Analog
\end{enumerate}

\section{Definition Monoid:}

Ein (assoziatives) Monoid $M$ ist eine Menge $M$ zusammen mit einer (bin"aren) Verkn"upfung 
$M \times M \to M: (x,y) \mapsto xy$ sodass folgende Axiome erf"ullt sind:

\begin{enumerate}[({M}1)]
\item Assoziativit"at: $\forall x,y,z \in M: (x \circ y) \circ z = x \circ ( y \circ z)$
\item neutrales Element: $\exists e \in M$ mit $e \circ x = x \circ e = x \forall x\in M$
\end{enumerate}

\section{Bemerkung}

\begin{enumerate}[(a)]
%TODO hier fehlt was. "Nur m1 \to M Halbgruppe (M = assoziative UG)" Oder so
\item H"aufig werden Gruppen (Monoide) mit einer zus"atzlichen Strukut versehen, etwa mit einer Topologie
oder mit einer differenzierbaren Strukutur ($\to$ Topologische Gruppe, Mannigfaltigkeit, Lie-Gruppe)

\item Ist $M$ Monoid und x rechts und links invertierbar, so ist x bereits invertierbar. 
Sind alle Elemente von $M$ invertierbar, so ist M bereits eine Gruppe.
\end{enumerate}

\section{Beispiele:}
\begin{enumerate}[(a)]
\item $X$ Menge, $M = End(X) = \{f: X\to X Abbildung\}$ mit Komposition ist Monoid, aber keine Gruppe.
\item $(\mathbb{N},+)$ Monoid
\item $(\mathbb{R},*)$ Monoid
\item Ist $M$ Monoid, so ist $M^* = \{x \in M, x$ invbar $\}$ bereits eine Gruppe.
\end{enumerate}

\section{Definition Gruppenhomomorphismus}
Seien $(G,+),(H,*)$ zwei Gruppen und $f: G \to H$ eine Abbildung. $f$ heist Gruppenhomomorphismus oder multiplikativ/additiv 
oder Strukturerhaltend oder mit der Algebraischen Strukut vertr"aglich,
wenn $f(x+y) = f(x)*f(y) \forall x,y \in G$

\section{Definition Untergruppe}

Eine Teilmenge $H$ einer Gruppe $G$ heisst Untergruppe wenn gilt:

\begin{enumerate}[({UG}1)]
\item Verkn"upfung von $G$ induziert Verkn"upfung $H \times H \to H auf H$. (man sagt: $H$ ist unter der Verkn"upfung von $G$ abgeschlossen)
\item $1_G \in H $ (dann ist automatisch $1_G$ neutral in H)
\item $\forall x \in H$ ist $x^{-1} \in G $ bereits in $H$. (man sagt: $H$ ist abgeschlossen unter Inversenbildung)
\end{enumerate}

\section{Beispiel:}
\begin{enumerate}[(1)]
\item $(\{1_G\}, *) \subseteq (G, *)$ triviale Untergruppe
\item $(G, *) \subseteq (G,*)$ %TODO das macht irgendwie nicht viel Sinn. Abschreibfehler?
\item $(\mathbb{Z}, +) \subseteq (\mathbb{Q}, +) \subseteq (\mathbb{R}, +) \subseteq (\mathbb{C}, +)$ aufsteigende Kette von UG
\item $SL_n(\mathbb{R}) \subseteq GL_n(\mathbb{R}) \subseteq (GL_n\mathbb{C}, +)$
\item $C^\infty Diffeome(\mathbb{R}, \mathbb{R}) = \{f: \mathbb{R} \to \mathbb{R}, C^\infty, bijektiv, f^{-1} C^\infty\} \subseteq$ Stetige Bijekionen $(\mathbb{R}, \mathbb{R})$
$\subseteq$ Bijektionen $(\mathbb{R}, \mathbb{R})$
\item $(\mathbb{Q}^* , *) \subseteq (\mathbb{R}^*, *) \subseteq (\mathbb{C}^*, *)$
\item $k,m \in \mathbb{N} (\sqrt[k]{1}, *) \mathrel{\widehat{=}}$ komplexe k-te Einheitswurzeln $ \subseteq (\sqrt[km]{1},*)$
\item $Diag_n(K) = \{\lambda 1_n , \lambda \in K^*\} \subseteq GL_n(K)$
\end{enumerate}

\section{Definition Normale Untergruppe}
Sei $H\subseteq G$ Untergruppe. Dann heisst $H$ normal in $G$ oder Normalteiler geschrieben $H \trianglelefteq G$, wenn $H$ unter Konjugation mit Elementen aus $G$ abgeschlossen ist.
$\forall x \in G : xHx^{-1} \subseteq H$ wobei $xHx^{-1} = \{xyx^{-1}, y \in H\}$ "Aquivalent dazu: $\forall x \in G: xHx^{-1} = H$ oder
$\forall x \in G : xH = Hx$

\section{Beispiele:}
\begin{enumerate}[(1)]
\item Jede Untergruppe einer kommutativen Gruppe ist normal.
\item $1,G \subseteq G$ sind normal.
\item $\{\lambda * 1_n, \lambda \in K^* \} \subseteq GL_n(K)$
Es gilt $\forall A \in GL_n(K) : (\lambda 1_n) A = A*(\lambda 1_n) (!)$
%TODO hier fehlt was "und und ist H \subseteq G normal" ? 
\item $H \subseteq GL_n(K) \cong GL(K^n)$ Dann ist $H$ normal, wenn $H$ unabhängig von der Wahl einer Basis in $K^n$ definiert werden kann.
$H$ normal: $\forall A \ in H \forall p \in GL_n(K) PAP^{-1} \in H$
\item $S_n = Bij(\{1, ... ,n\})$ Permutationsgruppe. Sei $\sigma, \tau \in S_n$ 
Dann ist $\tau \sigma \tau ^{-1}$ jene Permutation, die man aus $\sigma$ erhält, wenn man die Zahlen $(1, ... ,n)$ mit $\tau$ permutiert (finde Beispiele).
%TODO Diagramm einfügen
Sei $H \subseteq S_n$ die Untergruppe jener Permutationen $\sigma \in S_n$ mit $\sigma (1) = 1$ (ist Untergruppe(!)).
Diese Untergruppe ist nicht normal, da sie von der Nummerierung der Zahlen 1...n abhängt.
\end{enumerate}

\section{Bemerkung}
Intuitiv bedeutet normale Untergruppe, dass die Definition unabhängig von der Bennenung der Elemente von $G$ definiert werden kann.

\section{Definition Faktorgruppe}
Sei $G$ Gruppe und $H \trianglelefteq G$ normale Untergruppe. 
Es bezeichnet $G/H := \{xH, x\in G\}$ mit $xH=\{xy, y\in H\}$ die Menge der Rechtsnebenklassen von G modulo H, geschrieben $xH = \bar{x} = [x]$
(das sind "Aquivalenzklassen von Elementen $x$ von $G$ bzgl. der "Aquivalenzrelation $x \sim z \Leftrightarrow xH=zH (!)$)
Auf $G/H$ wird durch $(xH)(zH) = (xzH)$ oder auch geschrieben $\bar{x} *\bar{z} = \bar{xz}$, eine Gruppenstruktur definiert.
$G/H$ mit dieser Verknüpfung heisst Faktorgruppe von $G$ nach $H$.

\section{Bemerkung:}
\begin{enumerate}[(a)]
\item Diese Konstruktion funktioniert nur für normale Untergruppen.
\item Ist $G$ kommutativ so ist $G/H$ definiert für alle Untergruppen $H$ von $G$.
\item Ist die Gruppe $G$ additiv so schreiben wir $\bar{x} = x+H = H+x$. 
Beispielsweise $(\mathbb{Z}, +)$ mit Restklassengruppe $\mathbb{Z}_K = \mathbb{Z} / K \mathbb{Z} = a+K\mathbb{Z} = \bar{a}$
\end{enumerate}

\section{Beweis:}
\begin{enumerate}[(1)]
\item Multiplikation auf $G/H$ ist wohldefiniert, d.h. unabh"angig von der Wahl des Repr"asentanten
Seien $x,x',z,z' \in G$ mit $xH=x'H$ und $zH=z'H$. Dann ist zu zeigen: $xzH = x'z'H$ Aber $xzH = x(zH) = x(z'H) = x(Hz') = xHz' = (xH)z' = (x'H)z' = Hx'z' = x'z'H$
\item Verkn"upfung assoziativ einfach
\item neutrales Element von $G/H$: $1_{G/H}= 1_gH =\bar{1}_G$. Zeige, dass dies das neutrale Element ist.
\item Inverses Element $zxH=\bar{x} (xH)^{-1} = x^{-1}H=\bar{x^{-1}}$ %TODO Hier stimmt irgendwas nicht
\end{enumerate}

\section{Beispiel:}
%TODO irgendwie unvollständig oder nicht so sehr verständlich?
$A_n = \{$alternierende Permutationen$\} \subseteq S_n$ ist Normalteiler, dann hat $\sigma \in S_n$ $sgn=1$, so auch $\tau \sigma^{-1},$ da $sgn$ multiplikativ ist.
Damit ist $S_n/A_n$ Faktorgruppe.  (man zeigt leicht: $S_n/A_n \cong (\{\pm 1\},*)$)

\section{Erinnerung}
$G,H$ Gruppen $f: G\to H$ Abb. heisst Gruppenhomomorphismus/multiplikativ wenn
\begin{enumerate}[({GH}1)]
\item $f(1_G) = 1_H$
\item $f(xy) = f(x)f(y) \forall x,y \in G$
\end{enumerate}
Die Verknüpfungen können auf $G$ und $H$ ganz verschieden sein, etwa 
$f:$ \mbox{$(\mathbb{R},+) \to (\mathbb{R}^*,*);$} \mbox{$x \mapsto exp(x) = e^x$} $f(x+y) = f(x) * f(y)$

\section{Definition Kern und Bild}
Ist $f: G \to H$ Gruppenhomomorphismus, so nennen wir $\{x \in G, f(x) = 1_H\}$ den Kern von f oder $kern(f)$ und
$\{y \in H, \exists x \in G : f(x) = y\}$ das Bild von F oder $Im(f)$.

\section{Satz}
Sei $f: G \to H$ Grupenhomomorphismus. Dann gilt:
\begin{enumerate}[(a)]
\item $kern(f) \trianglelefteq G$ normale Untergruppe
\item $Im(f) \leq H$ nur Untergruppe (im allgemeinen nicht normal)
\end{enumerate}

\section{Beweis}
Siehe Proseminar. Hier: $kern(f)$ ist normal.

Sei also $x \in kern(f)$ und $y \in G$. Zu Zeigen: $yxy^{-1}\in kern(f)$. Aber $f(yxy^{-1}) = f(y)f(x)f(y^{-1}) = f(y) 1_H f(y^-1) = f(y)f(y^{-1}) = f(y)f(y)^{-1} = 1_H$

\section{Homomorphiesatz für Gruppen}
Sei $f: G \to H$ Gruppenhomomorphismus. Dann ist die induzierte Abbildung $\bar{f}: G/kern(f) \to Im(f); \bar{x} \mapsto f(x)$
ein wohldefinierter Gruppenisomorphismus mit zugehörigem kommutativem Diagramm:
%TODO hier kommutatives Diagramm einfügen

\subsection{Bemerkung}
Die kanonische Abbildung $\pi$ $G \to G/kern(f) $ ist ein surjektiver Gruppenhomomorphismus.

\subsection{Beweis}
\begin{enumerate}[(a)]
\item Wohldefiniert: Zu Zeigen: $\bar{x} = \bar{z} \Rightarrow \bar{f}(\bar{x}) = f(x) = f(z) = \bar{f}(\bar{z})$ $\forall x,z \in G$
Setze $ K = kern(f)$ dann $ \bar(x) = \bar{z} \Leftrightarrow  xK = zK $ Insbesondere $ \exists y \in K mit x = zy$. 
Damit ist $f(x) = f(zy) = f(z)f(y) = f(z) 1_H = f(z)$

\item $\bar{f}$ ist Gruppenhomomorphismus. Selbst nachprüfen(!)

\item $\bar{f}$ surjektiv. 
Sei $w \in Im(f)$ wähle $x \in G$ mit $f(x) = w$. 
Dann gilt $\bar{f}(\bar{x}) = f(x) = w$, also ist $f$ surjektiv.

\end{enumerate}
\section{Definition}
%TODO <s> besser machen (< und > näher an s)
%TODO Betragsstriche besser machen
Sei $G$ eine Gruppe, $s \in G$. 
Dann bezeichnet $<s>$ die von $s$ erzeugte Untergruppe von G. 
$<s> = \{s^k | k \in \mathbb{Z}\} \subseteq G$ 
(Dies ist wirklich eine Untergruppe von G aufgrund. $s^i + s^j = s^{ij} $und$ s^{-i} \in <s>$)
$G$ heisst zyklisch falls $\exists s \in G: <\!s\!>=G$
Wir setzen $ord_G(s) = |<\!s\!>_G|$

\subsection{Bemerkung}
Gilt $ord_G(s) = k \Rightarrow s^k = 1_G$
Es ist n"amlich $<s> = \{1,s,s^2, ... , s^{k-1}\}$
%TODO hier fehlt etwas
$s^k = s^l mit 0 \geq ??? \geq ???$

\subsection{Beispiel}
$(\mathbb{Z}, +): \mathbb{Z} = \{k*1 | k \in \mathbb{Z} \} = <1>$
$(\mathbb{Z} / n\mathbb{Z} , +) \mathbb{Z} / n\mathbb{Z} = \{k (1^+ n \mathbb{Z}) | k \in \mathbb{Z}\}$

\section{Satz}
\begin{enumerate}[(1)]
\item Alle Untergruppen von $\mathbb{Z}$  sind von der Form 
$n\mathbb{Z}$ f"ur $n \in \mathbb{N} \cup \inf$
\item Alle Faktorgruppen von $\mathbb{Z}$ sind von der Form 
$\mathbb{Z} / n\mathbb{Z}, n \in \mathbb{N} \cup \{0\}$
\item Alle zyklischen Gruppen sind isomorph zu $\mathbb{Z}$ oder $\mathbb{Z}_n$ verm"oge:
\begin{enumerate}[]
\item $ f: \mathbb{Z} \to G; k \mapsto s^k $ wobei $G = <s>$ und $|G| = \inf$
\item $ \bar{f}: \mathbb{Z}/n\mathbb{Z} \to G ; \bar{k} \mapsto s^k$ 
wobei $G = <s>$ und $|G| = n$
\end{enumerate}
\end{enumerate}

\subsection{Beweis}
\begin{enumerate}[(1)]
\item Sei $H$ eine Untergruppe von $\mathbb{Z}$.
Falls $H = \{0\}$, so gilt $H = 0\mathbb{Z}$.
Falls $H \neq \{0\}$, so enthält %hier fehlt was%
mindestens ein $a \in H$ mit $a \neq 0$ o.B.d.A $a > 0$, den falls a $a < 0$,
so liegt $-a \in H$ und $-a > 0$. 
$H$ enhält ein kleinstes positives Element, welches wir mit $a$ o.B.d.A. schon gewählt haben.
Behauptung: $H = a\mathbb{Z}$. Klarerweise gilt $a \in \mathbb{Z} \subseteq H$
Sei $h \in H$. wir müssen zeigen $\exists q \in \mathbb{Z}: h = q*a$
Wende die Division mit Rest auf $(h,a)$ an: h = q*a+r mit $q \in \mathbb{Z}, 0\leq r < a$
Es ist $h,a \in H \to h-qa = r \in H.$
Wegen der Minimalität von a muss $r = 0$ gelten.
damit $h = q*a$ und weil $h \in H$ beliebig, folgt $H \subseteq a\mathbb{Z}$
\item Aus $(1)$ folgt das alle ELemente %Hier fehlt was.
\item Sei $G$ eine zyklische Gruppe, $G = <s>$. 
$f: \mathbb{Z} \to G ; k \mapsto s^k$ 
Dies ist ein Gruppenhomomorphismus von $(\mathbb{Z},+) \to (G,*)$ und surjektiv.
Sei $H := kern(f)$. Dann ist $H = n\mathbb{Z}$ für ein $n \in \mathbb{N} \cup \{0\}$
\end{enumerate}

\subsection{Korollar:}
\begin{enumerate}[1)]
\item Jede Untergruppe einer zyklischen Gruppe ist zyklisch.
\item Falls $G \cong \mathbb{Z}$ und $G = <s>$, 
dann gilt $H = <s^m>$ für ein $m \in \mathbb{N} \cup \{0\}$
\item Falls $G \cong \mathbb{Z}/n\mathbb{Z}$ für ein $n \in \mathbb{N}$ und $G = <s>$,
dann gilt $H = <s^m>$ mit $m|n$
\end{enumerate}
\subsection{Beweis}
$G \cong \mathbb{Z}$. Dann ist $H \cong m\mathbb{Z}$ für ein $m \in \mathbb{N} \cup \{0\}$
und diese ist zyklisch $(H=<mk|k\in\mathbb{Z}>)$
$G = \mathbb{Z}/n\mathbb{Z}$. Dann ist $|G|=n$. Ist $G=<s>$, so ist $s^n = 1_G$
Für $H \leq G$ (H Untergruppe von G), $h,k \in H$ mit $h=s^a$, $k = s^b$,
so enthält $H$ auch $s^{ggT(a,b)}$
Sind $h_1, ..., h_n$ alle Elemente von $H$, mit $H_i = s^{ai}$, 
so gilt $S^{ggT(a_1,...,a_r)} \in H$ und sogar $<s^{ggT(a_1,...,a_r)}> = H$. 
Wegen $s^n = 1_G = 1_H \in H$ ist $n \in \{a_1, ... , a_r\}$ 
und somit $ m = ggT(a_1, ... , a_r) | n$.
Sei $G$ eine beliebige Gruppe (endlich)

\subsection{Lemma}
Sei $a \in G$ mit $ord_G(a) = n$, dann gilt $ord_G(at) = \frac{n}{ggT(n,t)}$

\subsection{Beweis}
Sei $d = ggT(n,t)$ und schreibe $n = dn'$ und $ggT(n',t') = 1$
Sei $r = ord_G(t)$ %eventuell Fehler hier?
Dann gilt: $1_G = \binom{t}{a}^r = a^{tr}$ und somit $tr = mn$ für ein $m \in \mathbb{N}$.
Also ist $dt'r = dmn'$ und damit $t'r = mn'$. 
Damit erhalten wir $n'/t'r$ und wegen $(n', t') = 1$ folgt $n'|r$, insbesondere $n' \leq r$
Andererseits gilt $(a^t)^{n'} = a^{dt'n'} = {a^n}^{t'} = 1_G$ und somit $r \leq n'$
Zusammen ergibt sich $n' - r$, d.h.: $ord_g(a^t) = n = n' = \frac{n}{d}$

\section{Satz (Lagrange)}
Sei $G$ eine endliche Gruppe, $H \leq G$ Untergruppe. Dann gilt $|H| | |G|$%TODO besser machen
\subsection{Beweis}
$G = \bigcup{x\in G}{xH}$ Falls $xH \cap yH \neq \emptyset$, 
dann existieren $a,b \in H$ mit $xa = yb$.
Daraus folgt $xaH = ybH \Rightarrow xH=yH$
Es gilt $|xH| = |H|$ (denn Multiplikation mit $x \in G$ ist bijektiv) und daher
$G = \bigcup{\bar{x} \in G}{\bar{x} H} \Rightarrow |G| = |G/H| * |H|$

\subsection{Definition}
Sei $G$ eine Gruppe, $S \subseteq G$ eine Teilmenge.
Dann heisst $<s> = \bigcap{s \leq G}{H}$ die von $S$ erzeugte Untegruppe von $G$.
(Damit die Definition gerechtfertigt ist, müssen wir nachweisen, dass der Durchschnitt beliebig vieler Untergruppen einer Gruppe wieder eine Untergruppe ist.)
$Hi \leq G$ für $i \in I \Rightarrow \bigcap{i\in I}{Hi} \leq G$
$1_G \subseteq Hi$ 
$\forall i \Rightarrow 1_G \in \bigcap{i\in I}{Hi}$
Sind $x,y \in \bigcap{i\in I} Hi$, so gilt $x \in Hi, y \in Hi, \forall i\in I $
und daher auch
$xy \in Hi, x^{-1} \in H, \forall i \in I$. 
Dann folgt
$xy, x^{-1} \in \bigcap{i\in I}{Hi}$
%Folgendes ist Lückenhaft
Exestiert eine Menge $S$ mit $|S| < \inf$ f"ur die $G=<s>$, so [...] endlich erzeugt.
Achtung: $G$ endlich $\Rightarrow$ $G$ endlich erzeugt

\subsection{Satz}
Sei $G$ Gruppe, $s \subseteq G$, $S \neq \empty$. 
Dann besteht $<s>$ aus allen endlichen Produkten von Elemente aus $S$ und $S^{-1}$ %???

\subsubsection{Beweis}
Sei $\bar{S} = \{a_1, ..., a_n ; a_i \in S \cup S^{-1}, n \in \mathbb{N} \}$
z.z. $\bar{s} = <s>$
\begin{enumerate}[]
\item $S \subseteq \bar{S}$
\item $\bar{s} \leq <s>$ denn f"ur jede UG $U$ von $G$ mit $S \leq U$ gilt:
$a,b \in U \Rightarrow ab \in U$ und $a^{-1} \in U$
Damit enh"alt $U$ alle endlichen Produkte von Elementen aus $S \cup S^{-1}$
Es folgt $<s> = \bigcap{s \leq U \leq G}{U}$ enth"alt $\bar{S}$
\item $<s> \leq \bar{s}$ Es gen"ugt daf"ur, $\bar{S}$ ist UG von $GG$ zu zeigen. 
$a,b, \in \bar{S}$
Schreibe $a= a_1, ..., a_n, b_1, ... , b_m$
Es folgt $ab^{-1} = a_1* ... * a_n* b_m^{-1} ... b_1^{-1} \in \bar{S}$
\end{enumerate}

\subsection{Korollar}
$\phi : <s> = G \to H$ sei ein Gruppenhomomorphismus. 
Dann ist $\phi$ durch $\phi/s$ eindeutig bestimmt.

\subsection{Beweis}
$a \in G a=a_1 ... a_n $ mit $a_i \in S \cup S^{-1}$
$\phi(a) = \phi(a_1) ... \phi(a_n) $ und falls $a_i \in S^{-1} \phi(a_i) = \phi(a_i^{-1})$


%Vorlesung 26.11

\section{Ringe:}

2 Verknüpfungen, meistens $+$ und $*$, mit Distributivgesetz. Wichtigstes Beispiel:
\begin{enumerate}[]
\item $(\mathbb{Z},+,*)$
\item $(K[x],+,*)$
\item $(M_n(K),+,*)$
\item $(\mathbb{Z}/m\mathbb{Z},+,*)$
\end{enumerate}

Dann Unterringe, Ringhomomorphismus, Ideal (spezielle Unterringe), nicht nur additiv abgeschlossen sondern auch unter Ring-Multiplikation

\subsection{Lemma}
$I \subseteq R$, dann gilt: $I$ Ideal in $R$ $\Rightleftarrow$
$\forall x,y \in I, \forall r \in R$ gilt: $x \pm y \in I$, $rx \in I$ (Insbesondere $xy \in I$)

\subsection{Bemerkung}
Ideal steht kurz für ideale Zahl beim Übergang $k \in \mathbb{Z}$ zur Idealen Zahl $k*\mathbb{Z} \subseteq \mathbb{Z}$
\subsection{Definition EZS}
Sei $I \subseteq R$ Ideal. Eine Teilmenge $X$ von $I$ heisst $R$-Erzeugendensystem (oder Erzeugendensystem von $I$ als Ideal) $\Rightleftarrow$
Jedes Element $y \in I$ ist endliche $R$-linearkombination von Elementen aus $X$ $\Rightleftarrow$ $\forall x \in I \eists x_1,...,x_k \in X \exists r_1,...,r_k \in R$
$y = \sum_{i=1}^k r_ix_i$ (Vergleich mit Untervektorraum). %TODO sicher Untervoektorraum?
$I$ heisst endlich erzeugt (als Ideal) $\Rightleftarrow \exists$ endliches EZS $X \subseteq I = <X>$ %TODO <X> besser machen

\subsection{Beispiele}
\begin{enumerate}[(1)]
\item $k \mathbb{Z} = \{k*a, a\in \mathbb{Z} \} \subseteq \mathbb{Z}$ Ideal mit EZS $\{K\}$ %TODO ist das wirklich so oder hab ich ich verschrieben?
$k = 0, k = \pm 1, \mathbb{Z} \subseteq \mathbb{Z}$
\item $R=K[x]$ Polynomringe in einer Variblen $x$ über Krper $K$. $I=\{p \in K[x], p(0)=0\} \subseteq K[x]$ Ideal
\item $R=K[x], I_a = \{p\in K[x], $Untergrad$(P) \geq a\} \subseteq K[x]$ Ideal ($a \in \mathbb{N}$)
\item $K$ Körper insbesondere auch Ring. Die einzigen Ideale sind $I=0$ und $I=K$.
\end{enumerate}

\subsection{Definition Summe, Durchschnitt, Produkt, Wurzel}
\begin{enumerate}[]
\item Es seien $I,J \subseteq R$ Ideale. Definiere: $I+J := <J,I> = $Erzeugnisse von $I$ und $J$ = kleinstes Ideal von $R$ das $I$ und $J$ enthält
= $\{x+y | x \in I, y \in J\}$
\item $I \cap J = \{x \in R, x \in I$ und $x \in J$ \} Durchschnitt von $I$ und $J$.
\item $I*J$ = $<x*y, x\in I, y \in J> = \{\sum{i=1}^k r_ix_iy_i, x_i \in I, y_i \in J, r_i \in R\}$
Produkt von $I$ mit $J$. Insbesondere: $I*I = I^2 = <x*y | x*y\in I>$
$I^k$ = $I*I*...*I*$ kte Potenz von $I$
\item $\sqrt{I} :=\{x\in R, \exists k\in \mathbb{N}: x^k \in I\}$ Wurzel oder Ideal von $I$.
\end{enumerate}

\subsection{Satz}
Diese Konstruktionen liefern Ideale von $R$. Beweis Übungsaufgae
\subsection{Beispiele}
%TODO Formatierung hiervon evtl schöner machen?
$R = \mathbb{Z}, I = 6\mathbb{Z}, J = 8\mathbb{Z}, I+J = 6\mathbb{Z} + 8\mathbb{Z} = 2 \mathbb{Z}$
$I \cap J = 6\mathbb{Z} \cap 8\mathbb{Z} = 24 \mathbb{Z}, I*J = 6\mathbb{Z}*8\mathbb{Z} = 48\mathbb{Z}$
$\sqrt{I} = 6\mathbb{Z} = I$ $\sqrt{J} = 2\mathbb{Z}$

\subsection{Satz}
$f: R\to S$ Ringhomomorphismus $I \subseteq R, J \subseteq S $ Ideale
\begin{enumerate}[(a)]
\item $f(I) \subseteq S$ Unterring aber nicht notwendigerweise Ideal
\item $f^{-1}(J) \subseteq R $ Ideal. Insbesondere $f(R) \subseteq S$ Unterring. $Kern(f) = f^{-1}(0) \subseteq R$ Ideal
\end{enumerate}

\subsection{Erinnerung:}
$k \in \mathbb{N}$ heisst Primzahl $\Rightleftarrow k \geq 2$ hat keine echten Teiler 
$\Rightleftarrow $teilt k ein Produkt $xy$, so teilt $k$ bereits einen der Faktoren 
$\Rightleftarrow$ Ist $xy \in k \mathbb{Z}$, so ist entweder $x \in k \mathbb{Z} oder y \in k \mathbb{Z}$

\subsection{Bemerkung}
$k,m \in \mathbb{Z}$ Dann ist $k$ ein Teiler von $m$ $\Rightleftarrow m\mathbb{Z} \subseteq k\mathbb{Zl \Rightleftarrow m \in k\mathbb{Z}$
Teilbarkeit entspricht des Inklusion von Idealen.

\subsection{Definition (prim, maximal, radikal)}
Es sei $I \subset R$ Ideal.
\begin{enumerate}[]
\item  $I$ heisst prim oder Primideal $\Rightleftarrow \forall x,y \in R mit x*y \in I$ gilt bereits $x \in I$ oder $y \in I$
$\Rightleftarrow$ Produkt liegt in $I$ wenn mindestens einer der Faktoren bereits in $I$ liegt.
\item $I$ heisst maximal oder maximales Ideal $\Rightleftarrow \forall J \subseteq R: I \subset J \Rightarrow J=R$
$\Rightleftarrow I$ besitzt kein Überideal $\neq R$.
\item $I$ heisst Radikal oder Wurzelideal $\Rightleftarrow \forall x \in R$ mit $x^m \in I$ für ein $m \in \mathbb{N}$ gibt es bereits x \in $I$
$\Rightleftarrow I=\sqrt{I}$
\end{enumerate}

\subsection{Beispiel}
%TODO Hier ist noch nicht fertig
\begin{enumerate}[]
\item $R = \mathbb{Z}$
\end{enumerate}

%Start 7.1.14


\section{Kapitel III Körpererweiterungen}
\subsection{Bemerkung}

Unterschied zu beliebigen Ring: In einem Körper sind alle Elemente ausser 0 multiplikativ invertierbar.

\subsection{Bezeichnung}
Ist $K$ Körper, so bezeichne $K[x] = \{ P(x) = \sum{i=0}^d a_i x^i , ai \in K, d\in \mathbb{N}$
ein Polynomring über $K$ in einer Variblen x mit x Symbol/Platzhalter.
Analog: $K[x_1, ..., x_n] = K[x_1, ... , x_{n-1}][x_n]$ Polynomring in $n$ Variablen.

\subsection{Definition (Körper)}
$K$ kommutativer Ring $\neq 0$ heisst Körper $\Rightleftarrow$ $K^* = \modulus{K}{\{0\}$ 
$\Rightleftarrow$ alle Elemente $\neq$ 0 von K sind invertierbar
(Schiefkörper, wenn nicht kommutativ)
\subsection{Beispiele}
\begin{enumerate}[]
\item $\mathbb{Q}, \mathbb{R}, \mathbb{C}, \modulus{\mathbb{Z}}{p\mathbb{Z}} = \mathbb{F}_p$
\item $\modulus{R}{m}, R$ Ring, $m \subseteq R$ maximales Ideal.
\end{enumerate}

Insbesondere: $K$ Körper, $K[x]$ HIR, $P \in K[x]$ irreduzibles Polynom $m = <p>$
von P erzeugtes Ideal ist Primideal und damit auch maximal (da $K[x]$ faktoriell),
also ist $L := \modulus{K[x]}{m}$ wieder Körper. 
Etwa $K = \mathbb{R}$, $P(x) = x^2 + 1$ irreduzibel über $\mathbb{R}$
$\modulus{\mathbb{R}[x]}{<x^2 + 1>} = \mathbb{R}[\bar{x}]$
$\bar{x}$ Restklasse von x erfüllt die Gleichung $\bar{x}^2 + 1 = 0$, also $\bar{x}^2 = -1$.
Damit $\mathbb{R}[\bar{x}] = \mathbb{R}[i] = \mathbb{C} = \mathbb{R} \bigoplus \mathbb{R}i$

\subsection{Notation}

In $\subsection{Q}$ kann man nicht aus allen positiven Zahlen Wurzeln ziehen,
zudem ist $\mathbb{Q}$ nicht vollständig, 
daher "Erweiterung" von $\mathbb{Q}$ auf $\mathbb{R}$ vollständig.
Um auch Wurzeln aus negativen Zahlen ziehen zu können, erweitert man nochmal von
$\mathbb{R}$ auf $\mathbb{C}$, vollständig und algebraisch abgeschlossen 
(Jedes (nicht konstante) Polynom in \mathbb{C}[x] hat eine Nullstelle in $\mathbb{C}$)
Also: $\mathbb{Q} \subseteq \mathbb{R} \subseteq \mathbb{C}$ Inklusionen von Körpern.

\subsection{Definition (Körpererweiterung)}
Sei $L$ ein Körper und $K \subseteq L$ ein Unterring 
(also mit er von $L$ induzierten Verknüpfung ist $K$ ein Ring)
$K \subseteq L$ heisst Körpererweiterung $\Rightleftarrow$ 
$K$ ist Körper, also bzgl. Inversionsbildung abgeschlossen (mit Ausnahme von 0)
$\Rightleftarrow$ $K \subseteq L$ ist Unterkörper $\Rightleftarrow$ $L$ ist Oberkörper von $K$.

\subsection{Bemerkung}
Ist $K \subseteq L$ Körpererweiterung, so ist $L$ auch $K$-Vektorraum, 
also ist die $K$-Dimension von $L$ definiert. 

\subsection{Definition Grad der Körpererweiterung}

Ist $K \subseteq L$ Körpererweiterung, so heisst $[L:K] := dim_KL$ der Grad der Körpererweiterung.
Die Körpererweiterung heisst endlich $\Rightleftarrow [L:K] < \inf$

\subsection{Beispiele}
\begin{enumerate}[]
\item $[\mathbb{R}:\mathbb{Q}] = \inf$ (ohne Beweis)
\item $[\mathbb{C}:\mathbb{R}] = 2$, denn $\mathbb{C} = \mathbb{R} + \mathbb{R}i$
\item Sei $P(x) \in K[x]$ irreduzibel von Grad $d$. 
Setze $L = \modulus{K[x]}{<P>}$. Körper mit $K \subseteq L$(!), also Körperweiterung. 
Aber nach euklidischer Division gilt: 
$L = \modulus{K[x]}{<P>} \cong K \bigoplus K\bar{x} \bigoplus K\bar{x}^2 ... \bigoplus K\bar{x}^{d-1}$
Also $[L:K] = d$
\end{enumerate}

\subsection{Satz}
Seien $K \subseteq L \subseteq M$ zwei Körperweiterungen (eigentlich drei: $K \subseteq L, L \subseteq M, K \subseteq M$).
Dann gilt: $[M:K] = [M:L] * [L:K]$

\subsection{Beweisidee}

Sei $[M:K] < \inf$. Wähle $K$-Basis $x_1, ..., x_l$ von $L$ und $L$-Basis $y_1, ... y_m$ von $M$.
Zeige dann, dass die Produkte $x_iy_j$ für $1 \leq i \leq l$, $1 \leq j \leq m$ 
eine $K$-Basis von $M$ bildet (EZS und K-linear unabhängig).

\subsection{Definition (algebraische Elemente)}

Sei $K \subseteq L $ Körpererweiterug und $\alpha \in L$ ein Element.
$\alpha$ heisst algebraisch über $K$ (oder algebraisch abhängig über $K$)
$\Rightleftarrow$ $\exists$ Polynom $P(x) \in K[x], P \neq 0$ mit $\alpha$ Nullstelle von P,
also $P(\alpha) = 0$. (Einsetzen ist wohldefiniert, da $\alpha$ in Körper $L$ liegt.)
Ist $\alpha$ nicht algebraisch über $K$, so heisst $\alpha$ tranzendent über $K$.
Elemente $\alpha_1, ... \alpha_n \in L$ heissen algebraisch unabhängig über $K$ 
$\Rightleftarrow$ es exestiert kein Polynom $P(x_1, ... x_n) \in K[x_1, ... x_b]$
ungleich Null mit $P(\alpha_1, ... \alpha_n) = 0$ (vgl. $K$-linear unabhängig mit $P$ linear).

\subsection{Beispiele}
$\pi \in \mathbb{R}$ ist tranzendent über $\mathbb{Q}$.
($\pi$ erfüllt keine polynomische Gleichung mit Koeffizienten in $\mathbb{Q}$)(Lindemann)
$e \in \mathbb{R}$ ist tranzendent über $\mathbb{Q}$

\end{document}

